\documentclass[dvipdfmx]{jsarticle}
\usepackage{bm}
\usepackage[ppl]{mathcomp}
\usepackage{array}
\usepackage{mathtools}
\usepackage{amsmath,amssymb}
\usepackage[top=20truemm, bottom=20truemm, left=20truemm, right=20truemm]{geometry}
\usepackage[dvipdfmx]{graphicx}
\usepackage{subcaption}
\usepackage{cases}
\usepackage{url}
\usepackage{physics}
\usepackage{graphicx}
\graphicspath{{./res/}}
\begin{document}

% タイトル
\title{CanSat の電装開発}
\author{FTE 9期 \quad 鄭潤賢}
\date{}         % 必要であれば記載
\maketitle

\tableofcontents
\clearpage

%%%%%%%%%%%%%%%%%%%%%%%%%%%%%%%%%%%%%%%%%%%%%%%%%%%%%%%%%%%%%%%%%%%%%%%%%%%%%%%%%%%%%
%%%%%%%%%%%%%%%%%%%%%%%%%%%%%%%%%% SEC 1 %%%%%%%%%%%%%%%%%%%%%%%%%%%%%%%%%%%%%%%%%%%%
%%%%%%%%%%%%%%%%%%%%%%%%%%%%%%%%%%%%%%%%%%%%%%%%%%%%%%%%%%%%%%%%%%%%%%%%%%%%%%%%%%%%%

\section{はじめに}
% あとで書く

\subsection{本書の構成}

\subsection{読者対象}

\subsection{感じてほしいこと}


\clearpage
%%%%%%%%%%%%%%%%%%%%%%%%%%%%%%%%%%%%%%%%%%%%%%%%%%%%%%%%%%%%%%%%%%%%%%%%%%%%%%%%%%%%%
%%%%%%%%%%%%%%%%%%%%%%%%%%%%%%%%%% SEC 2 %%%%%%%%%%%%%%%%%%%%%%%%%%%%%%%%%%%%%%%%%%%%
%%%%%%%%%%%%%%%%%%%%%%%%%%%%%%%%%%%%%%%%%%%%%%%%%%%%%%%%%%%%%%%%%%%%%%%%%%%%%%%%%%%%%

\section{CanSat を作り上げるもの達}

\subsection{CanSat ってなんだ}
% 字面の定義ではなく,より具体的に,実体的に

CanSat とは何だろうか.ググれば答えは簡単に手に入る.CanSat でググれば,学生が作る擬似人工衛星だとか小型惑星探査機
といったような答えが手に入るだろう.もちろんこのような答えも間違いではないだろう.だが,いざ CanSat を作ろうとすると
このような説明はしっくりこない.なぜなら人工衛星の電装を作ったことのある人などほとんどいないからだ.もう少し CanSat 
というものを電装を作る工程がイメージ出来るような形に変形できないだろうか.

CanSat はもっと大雑把に言えば「宇宙開発を目的とした自律型ロボット」であると言えるだろう.この CanSat の説明でのポイントは
3つある.1つ目は宇宙開発を目的としている点である.つまり,製作される CanSat と呼ばれるものは宇宙開発のために使用されることを
想定して作られたものである.2つ目は自律型であるという点である.これは CanSat と呼ばれるものは誰かが操縦することで動作する
だけでなく,操縦者なしでも目的の動作をすることが出来るという意味である.最後の3つ目はロボットという点である.ロボットの定義は
様々だが,ここでは人の代わりに何らかの作業を行う機械という意味で使用した.これで我々は CanSat の電装はロボットの電装に似ている
ということがわかった.それではロボットの電装とはなんだろうか.

ロボットの電装を知るためにロボットを分解してみよう.例として人型ロボットを考えてみよう.人型ロボットは手足の関節を動かせる.
それならきっとロボット内部に関節を動かす何らかの装置があるに違いない.それはなんだろうか.最近の人型ロボットは人が近づくと
反応するものもある.それならきっとロボット内部にカメラか何かがあって,人が近づいたことを検知しているに違いない.果たして
どうやっているのだろうか.しゃべる人型ロボットも多い.それならきっとロボット内部にスピーカーが付いているに違いない.
だがその声の主はなんだろうか.ロボットを観察すればするほどロボットのある動作を引き起こす何らかの装置があることに気づき,
またその装置はとても巧妙に動作していることがわかってくる.そう.ロボットとはあらゆる装置を駆使してプログラムされた
総合的なからくりなのである.そう考えると必然的にロボットの電装にはそれらの装置が乗っかってくることがイメージ出来るだろう.
そしてあらゆる装置が乗っかった電装にはロボットを思い通りに動作させるための何かがあるはずだ.

さあ,CanSat の実体が少しずつわかってきた.まず CanSat というのはロボットの一種に過ぎない.そしてロボットの電装には
様々な装置が乗っかっており,それらを動かす何かがあるのだ.まず電装を理解する第一歩は,その装置が何で,装置を動かすものは
何なのかをはっきりさせることだ.本章ではその点をはっきりさせよう.

\subsection{役者のご紹介}
% 部品の紹介

本節では上述したロボットの装置やその装置を動かす実体について明らかにしていく.簡単に言うと CanSat で使う部品の紹介である.
ここで注目してほしいのは,使用する部品には必ず役割があるという点である.それぞれの役割はロボットの動作に直結する.逆に,
ロボットの特定の動作にはある部品が不可欠であることが逆算できる.この逆算を出来るようになることこそが,電装設計の第一歩
である.そのためにはまず「順算」が出来なければならない.

\subsubsection{CanSat の脳}
\label{subsub:CanSat の脳}
% シングルボードコンピュータ,マイコン

先程はロボットの装置を動かす何かがあると述べた.その何かがまさしくここで説明する部品たちである.CanSat の脳として
各部品を動かし制御するのは CPU を持つコンピュータの役目だ.コンピュータといっても一般に使われるパソコンのようなものでは
ない.よく使われるのはシングルボードコンピュータと呼ばれるものやマイクロコントローラと呼ばれるものだ.いずれも小型である
という点がポイントだ.

シングルボードコンピュータとはその名のとおり1つのボードに全てが収まったコンピュータである.その類のコンピュータはパソコンの
ようにディスプレイやキーボードなどは内蔵していない.その代わり小型であり,手のひらに収まる程度のサイズである.
シングルボードコンピュータの代表格は Raspberry Pi であり,最近では Jetson Nano などがある.シングルボードコンピュータは
後述するマイクロコントローラに比べて高性能である.CPU 性能も高ければメモリの容量も圧倒的に大きい.その反面,値段は
シングルボードコンピュータの方が高く,サイズも小型とは言え大きい.また,シングルボードコンピュータのアーキテクチャは基本的には
パソコンのものと似ているため,シングルボードコンピュータでは OS が動く (もっぱら Linux 系が多い).OS が動くため OS の機能を
利用したプログラミングが可能であるという点も魅力の1つである.

マイクロコントローラ (通称マイコン) は通常のパソコンとは方向性が若干異なるコンピュータである.パソコンでは主にデータの管理に
焦点を充てて処理をすることが多いが,マイコンは周辺機器の制御に焦点を充てて設計されている.このため,大量のデータを扱う必要もなく,
メモリの容量もパソコンやシングルボードコンピュータの一般的な容量に比べるとかなり劣る.また,特に OS がないものが多い.
これにより,マイコンは制御以外の無駄な処理をすることがない (OS があるものだとプロセスの実行は OS が管理するので,
所望のプロセスだけを集中的に実行するということは出来ない) .ただし,マイコンは OS がないので必然的に OS の機能を利用する
ことが出来ない.マイコンはシングルボードコンピュータに比べて安価に手に入ることも考慮すべき点である.

ここまで CanSat の脳となるコンピュータの種類を紹介してきたが,今度はコンピュータという装置についてもう少し踏み込んで考えてみよう.
読者のみなさんの中にはパソコンを購入しようとして販売店やサイトを漂いながら製品のスペック (仕様) というものを目にしたことがあるかもしれない.
このスペックには主に CPU やメモリの容量,ストレージの容量などが記述されている.なぜそのようなものを表記するのだろうか.その答えは簡単で,
それらはこれからパソコンを購入しようとする人に対してヒントを提示するためである.そのヒントの1つである CPU とは Central Processing Unit の略で
中央処理装置などと訳される.CPU の役目は演算であり,基本的には演算に使われるデータは持っていない (一時的なデータは少し持っている).
それではどこからそのデータが来るのか.その答えはメモリであり,ストレージでもある.もしかしたら,USB メモリのように外部ストレージからデータを
取り出す必要があるかもしれない.コンピュータ内部の細かい話を無視すれば,コンピュータがすることはメモリ (もしくは何らかのストレージ)にあるデータを
 CPU に移して,CPU でなんらかの演算をさせ,演算結果をまたメモリ (もしくは何らかのストレージ) に返す,ということを反復するだけである.
これは非常に大雑把な議論であるが,十分本質的であり,開発者が常にイメージすべきことである.この議論から,解決すべき問題は次の3つに帰着する.
\begin{enumerate}
  \item どこからデータを取ってくるのか
  \item データをどう演算するのか
  \item 演算したデータをどうするのか
\end{enumerate}
上記の問題の1,3の問題は本章で解決される.また2の問題は第5章で解決される.いずれにしても,上記の問題は電装を設計する上で極めて本質的な
問題であり,設計における出発点とも言えるものである.

最後にデータについて述べておこう.データとはなんだろうか.データとは情報のことであると答える人がいるかもしれない.では情報とはなんだろうか.
実はこの問題は非常に深遠な問題であり,筆者もその答えを知らない.ただ,現在のコンピュータにおけるデータについては述べることが出来る.単に
データといえば抽象的でわかりにくいが,データを扱う対象が定まれば具体的な実体が見えてくるのである.現在のコンピュータにおいてデータとは
電圧値である.全てはこの電圧が握っている.よくコンピュータ上のデータは1と0の羅列だという説明を見かけることがあるが,実際にコンピュータ内部の
どこに1や0があるというのだろうか.1か0は人間の頭の中に概念に過ぎないはずだ.実際にはコンピュータ内部に1や0があるわけではない.コンピュータを
開発した賢い先人たちが,この1や0という概念を物理的に表現する方法を考え,物理現象として1や0と対応する実体がコンピュータ内部にあるのである.
その実体こそが電圧値ということだ.そしてその電圧値はとても雑に扱われる.高いか低いかだ.高ければ1を表し,低ければ0を表す (逆でも良い),といった
具合に対応付けられるのである.データを維持するためにはこの電圧値を維持しておけばいいわけだ (その役目がメモリだ).それではこのデータを他の場所に
移すにはどうすれば良いのだろうか.これまでの議論から,データを移すことはある場所にある電圧値を他の場所に移すことに対応することがわかるだろう.
そっくりそのまま移してもいいし,一度変形して移して受け取り側で復元してもいい.あとは技術や方法の問題だ.基板に配線をして金属のなかを通して
データを輸送する場合もあれば,データを変形し電波として送り受け取り側で復元する方法もある.このようなデータのやり取りについての具体例は
後にいくつか扱う予定である.しかしここで最も押さえておくべきことはデータを表すには物理的な実体を必要とすること,そしてそのデータのやり取りも
また物理現象であるということである.そのことさえ理解していれば,どんなデータ表現ややり取りの方法に遭遇しても上手く対処できるだろう.

% 表,図

\subsubsection{情報をくれるやつ}
\label{subsub:情報をくれるやつ}
% センサ,通信機器

前項では CanSat の脳に相当する部分を説明した.そしてコンピュータについて深堀りし,コンピュータを扱う上ではっきりさせなければならない3つの
問題を提示した.本項ではその1つ目の問題である「どこからデータを取ってくるのか」という問題について議論する.その上で,読者は CanSat を
構築する上で非常に重要な構成部品について学ぶことになる.より理解を深めるために前項の内容を常に意識しながら本項に臨んでいただきたい.

前項で「どこからデータを取ってくるのか」という問題を提示した.そしてデータとは実際にはどういうものなのかについて説明した.ここでまた
疑問が浮かぶ.一体全体データの出発点はどこなのだろうか.少し具体例を考えてみよう.読者はタイヤの付いたローバー型の CanSat を作って,
特定のゴールへ向かって走らせたい.さて,特定のゴールの位置の情報はどこから来るのだろうか.ゴールの位置はわかっても自分の位置がわからなければ
ゴールの位置に近づいていることもわからないはずだ.では自分の位置の情報はどこから来るのだろうか.声を掛けてやろうか.あいにく CanSat は
言語を理解しなかった.困った困った.

上述した例からわかったことは,どこからデータを取ってくるのかという問題を解決するためには,データの出発点をまず押さえておく必要があるという
ことだ.これは当然のことではあるが,やはり本質的なことである.先人たちはこのような問題を解決するために,データを生み出す方法を考えた.
ある人たちは自然からデータを作り出す方法を考え,ある人たちは既存のデータを輸送する方法を考えた.いずれにしても物理現象を上手く利用した
方法である.

自然からデータを作り出す装置はセンサと呼ばれる.多くのセンサは物理現象を何らかの方法で電圧値に変換し,その電圧値をコンピュータと交換する
手段を持っている.コンピュータの発達していなかった昔ながらのセンサにはアナログ電流計や圧力計などがある (これは物理現象を針の振れ度合いに
変換し人間に視覚的に伝えてくれている).我々はコンピュータを用いてデータを演算したいので,ひとまず対象とするセンサはデータをコンピュータと
交換することが出来るものに限定しよう.センサには様々な種類があり,それぞれ対象とする物理量を持っている.センサが作り出した物理量のデータは
コンピュータへ送られ,コンピュータはその物理量 (あるいは何らかの変換を施したもの) を使って CanSat の状態を把握することが出来るという仕組みだ
\footnote{CanSat の状態は,センサの作り出したいくつかの物理量を変数とする何らかの関数であるという見方も出来る.ただし注意が必要なのは,
もれなく全ての変数を見つけること,そして見つけたとしても状態関数の具体形を書き下すことは極めて困難であるということだ.例えば複数のセンサを使って
9種類の物理量のデータを作り出し,それを状態関数の変数に採用した場合,この9次元空間 (状態空間) 内のある1点が状態に対応することになる.果たしてそのような
高次元な空間内の関数を解析して,状態を正確に把握することは可能だろうか.残念ながらこれは容易ではない.これを実現する技術として機械学習,
特にディープラーニングを使った学習モデルを利用して解析を行う方法なども近年は発展目覚ましいが,まだまだ CanSat のコンピュータの処理能力と比べて
負荷が大きい部分がある (これもまた近年改善されてきているのだが).いずれにしても,状態空間全体を取り扱おうとするには高度な技術が必要だ.
通常はケースに応じて人為的に次元を落として状態を決定する.例えば,暗闇から出たことを照度センサのみ (つまり1次元状態空間) を使って判定するなどである.
付随する話題は5章,6章で議論する.}.表\ref{tab:2.2.2-sensors}は CanSat を製作する上でよく利用されるセンサの一覧である.

\begin{table}[htbp]
  \centering
  \caption{CanSat でよく用いるセンサの種類と概要}
  \begin{tabular}{c|r} \hline
    種類 & 物理量 \\ \hline
  \end{tabular}
  \label{tab:2.2.2-sensors}
\end{table}

CanSat が使用するデータは自然から抽出されたものだけではないはずだ.人間が CanSat に伝えたいことだってあるだろうし,他の CanSat が CanSat に
データをあげたいこともある.つまり通信だ.通信は他の地点にあったデータを別の地点に移動させることを指す.通信を行える装置は通信機器などと呼ばれ,
特に電磁波 (電波も電磁波の一種だ) を利用した通信機器は無線機と呼ばれたりする
\footnote{この業界の用語の定義は曖昧極まりないので微妙な違いをあまり気にしてはならない.用語の定義が曖昧になる理由としては,ベンダー達が
ワードパワーで自社製品を売りつけようと企み様々な造語を作成することで,結果的に本質的には似たようなデバイスや概念でありながら,名前の異なる名称が
生まれてしまったためと筆者は考えている.筆者の意見としては,小さな問題に引っかからないためには,細かいことよりも本質的な部分を理解することに
努めるべきだと思う.}.
通信というものはデータを送る装置とデータを受け取る装置があって初めて成立するが,一般的な通信機器はそれらを両方備えている場合がほとんどである.
我々がよく利用する通信機器には例えば PC やスマートフォンなどがある.ただしこれらの通信方法はとても複雑である.例えば,スマートフォンのブラウザ
アプリで検索をしてあるホームページを見ようとすると,そのときの通信は決してスマートフォンとホームページの HTML ファイルを送ってくるサーバー間だけの
通信ではない.スマートフォンから送信された電波は最寄りの基地局にたどり着き,基地局から局舎へ,局舎からバックボーンと呼ばれる通信網へと接続される.
海外サイトなら海底ケーブルを通ってデータが送信されるかもしれない.そしてサーバーにたどり着いたデータはサーバー側で適切に処理され,その結果が
スマートフォンに送られてくるのである.しかし,CanSat を製作する上でこのようなインフラを構築することは難しい.そこで,通常はデバイス間だけで
通信を行う通信モジュールを利用する (ブロードキャスト通信\footnote{他の端末へ一斉送信する通信方法.端末とはネットワークの終端点を意味する言葉である.}
をするものもある).CanSat でよく利用する通信モジュールを表\ref{tab:2.2.2-communication-modules}にまとめた.

\begin{table}[htbp]
  \centering
  \caption{CanSat でよく用いる通信モジュール}
  \begin{tabular}{c|c|r} \hline
    通信モジュール & 周波数帯 & 概要 \\ \hline
  \end{tabular}
  \label{tab:2.2.2-communication-modules}
\end{table}

実はセンサにも通信の要素が存在する.コンピュータとのデータのやりとりだ.データをやりとりするとはすなわち通信をすることなのである.
だからといって,センサに無線通信モジュールが組み込まれているわけではない.通信は金属内を通って行われる.したがって,センサとコンピュータを
適切に配線してあげる必要がある (詳細は2.4項,第4章で述べる).この方法は上述した無線通信モジュールの通信方法とは異なる方法だが,
れっきとした通信である.もっと言えば,通信という言葉は人間が勝手に作り出した頭の中だけで有効な概念なのであり,その概念を具体的に (つまり物理的に)
表現する方法が異なるだけの問題なのである.このような些細な問題よりも重要なことは,データをやりとりするためには通信という動作をする必要があり,
通信の実現方法はデバイス (装置) によって異なるということを知っておくことだ.このことさえ知っていれば,コンピュータの外部デバイスを利用するときに
それがコンピュータとどのように通信するのかを必ずチェックするようになるはずだ.

最後に荒業とも言える情報の伝え方を伝授しよう.その方法とは,キーボードをつなげてコンピュータに直接データを入力してしまうという方法である.
荒業ではあるがこれも立派な通信である.キーボードには押されると電気信号が発生するような装置が仕込まれており,発生した電気信号は有線のキーボードなら
その線を通って,Bluetooth 対応のキーボードなら Bluetooth 用の通信モジュールによって電波に変換されコンピュータに送信される.コンピュータは
受け取った信号を解析して文字を特定し,ファイルの指定された位置にその文字を書き込む.ただし,この方法は本番環境では通用しないことは読者の
みなさんも承知のとおりである.「こういう方法もあるということを忘れるな」ということを強調することが真の目的である.

さて,CanSat の電装 (ロボットの電装でもある) が段々と明るみになってきた.本項では「情報をくれるやつ」にはどのようなものがあるか,
そして情報が移動させることが通信であることを学んだ.これによって我々はコンピュータに情報をあげることが出来る役者の存在を知ることが出来た.
ここまでの流れを確認してみよう.コンピュータにはデータを使って演算をする,演算するためのデータをくれる役者を知った.これで演算は出来ることに
なるが,演算したらその後はどうするのだろうか.今のところその答えは出ていない.その詳しい議論は次項に譲ることにしよう.

% 表,図

\subsubsection{情報を欲しがるやつ}
\label{subsub:情報を欲しがるやつ}
% アクチュエータ,ストレージ

前項までで\ref{subsub:CanSat の脳}で提示した3つの問題点の1つ目が解決した.残す2つのうち,本項では3つ目の問題である「演算したデータをどうするのか」
について考えることにする.この問いは一見馬鹿馬鹿しく思えるが,よく考えてみるともっともな意見である.データを外部から取得してまでして演算をしたのに,
演算したデータをどうするのかが決まっていないなんてことはないだろう.では実際にそのデータをどうするのだろうか.答えは簡単だ.「情報を欲しがるやつ」に
あげればいいのだ.それでは「情報を欲しがるやつ」とは一体全体何者なのだろうか.もっとわかりやすく言えば情報を受け取ることで機能するデバイスとは何だろうかという
問いである.これに答えるためにはまずデータの使いみちを知る必要がある.使いみちが分かればそこには必ずデータを受け取るものがあり,それが答えとなるはず
だからだ.

それでは前項で見た具体例を再度考えてみよう.その具体例の状況は,読者がいまタイヤ付きのローバー型 CanSat を作って,特定のゴールへ向かって
走らせたいと考えているというものであった.前項で学んだことを活用すればゴール地点の位置情報はは事前にキーボードでゴール地点の緯度および経度を
直接入力してしまえば済むし,自身の位置データは GPS 受信モジュールなどを使って得ることが出来る.つまり,この時点で演算する準備は整っているわけだ.
そしてこの状態から例えば現在地からゴールまでの距離を演算によって得たとしよう.この距離のデータは何に使おうか.当然 CanSat の状態を決定するために
使いたい.それだけではなく,CanSat をゴールまで移動させるためにも使いたいはずだ.簡単のため CanSat はゴール方向を向いているとして話を進めると,
ゴールまでの距離が残り100mもあるのなら,CanSat を直進させるべきであろう.また,ゴールまでの距離が0.1m程度であればかなり十分と言える距離まで
近づいているので場合によっては停止してもいいかもしれない.つまり,演算して新しく生成されたデータ (もしくはセンサなどから得られた生データ) は
CanSat を次の状態へと変化させるために使われる可能性がある.

データの使いみちはまだある.データを人に見せることだ.コンピュータが受け取ったデータ,そのデータから新たに作り出したデータを人間が見たい場合は
多々ある.これらのデータは今後の開発や CanSat の動作をより深く理解するためには有益であるし,データを見たい人たちに提供することでビジネスに
なったりする.データを人に見せることというのは案外価値のあるものなのだ.

さて,データの使いみちがわかった.データを受け取る装置を整理しよう.まず CanSat の状態 (主に位置や姿勢) を変化させるために使う装置は
アクチュエータと呼ばれものである.アクチュエータとは入力されたデータに対し物理的な運動によって応答をする装置である.アクチュエータの
代表格としてはモータ (電動機) と呼ばれるものがあり,モータは電磁誘導とローレンツ力を組み合わせて電気的なエネルギーを力学的なエネルギーへと
変換する.モータには様々な種類があり,用途によって使い分けることが重要である.CanSat でよく利用されるモータを表\ref{tab:2.2.3-motors}に示す.

\begin{table}[htbp]
  \centering
  \caption{CanSat でよく用いるモータ}
  \begin{tabular}{c|r} \hline
    モータ & 概要 \\ \hline
  \end{tabular}
  \label{tab:2.2.3-motors}
\end{table}

アクチュエータは自由気ままに動くわけではない.アクチュエータには制御というものがつきまとう.アクチュエータはあるデータ (もっぱら電気信号) を
受け取り,そのデータにしたがって出力を出すようになっている.アクチュエータが暴走しないためには電気信号を適切に設定し,出力を意図したものに
近い形にする必要がある.このことを制御と呼ぶ\footnote{制御という言葉自体はアクチュエータに対してのみ使われるものではない.基本的に入力と
出力があるような対象に対して,出力を統制することを制御する言う.}.最終的に制御を決めるのは入力であるデータであることから,アクチュエータは
データを必要とする装置であると言える\footnote{ここではデータを概念的な情報と解釈するのではなく,物理的実体のある情報 (電圧,電流,磁気など) 
として捉えてほしい.}.また,アクチュエータによってはコンピュータからの直接の信号を受け付けないものも存在する.そのようなアクチュエータを
操作するためにはドライバと呼ばれる IC\footnote{Integrated Circuit の略.直訳すると集積回路.複雑な回路がチップのようなパッケージにまとめられて
いるものを指す.} を利用する必要が出てくる.モータに対するドライバはモータドライバと呼ばれ,モータの代わりにモータドライバが信号を受け付け,
モータドライバはその入力をもとに適切な出力をモータへ送る.

データは人に見せるために使われるとも述べた.そのためにはデータをとりあえず保管する必要があるかもしれない.データを保管しておくための装置は
ストレージと呼ばれる\footnote{もっぱら不揮発性メモリに対して使われることが多い.不揮発性とは電力の供給がなくともデータが消失しないという
性質を表す言葉である.}.ストレージにも様々な種類があり,これは読者のみなさんにも馴染み深いものも多いだろう.よく使われるストレージを表
\ref{tab:2.2.3-storage}に示す.

\begin{table}[htbp]
  \centering
  \caption{CanSat でよく用いるストレージ}
  \begin{tabular}{c|r} \hline
    ストレージ & 概要 \\ \hline
  \end{tabular}
  \label{tab:2.2.3-storage}
\end{table}

データを人に見せるために使う装置はまだある.代表的なのはディスプレイ (モニター) である.実際に見ているのはディスプレイ上に表示された
データであるが,そこに行き着くために無線通信モジュールなどの通信手段を取ったかもしれない.これもデータを受け取る装置である.もっと言えば,
演算を行う CPU もデータを受け取る装置だし,メモリも同様である.センサにも設定を書き換えるためにデータを受け取る仕組みがある.
もうここまで来るとデータを与えるだけのデバイスやデータを受け取るだけのデバイスはあまりないことに気がつくだろう.ここまで,便宜上
データを与えるものとデータを受け取るもので説明してきたが,ここで限界が来てしまった.だからといって焦ることはない.データを与える,
データを受け取る,そしてデータを処理するという動作は依然として保たれている.このうち複数の動作をするデバイスがあることがわかっただけだ.

ここまでで CanSat の電装に乗っかる多くの装置たちについて見てきた.そして本項でも継続的にデータの所在・行き先を強調しながら議論を進めてきた.
データがどこにあるのか,データがどこに行くのかというのはデバイス間の接続,そして回路を設計する上で非常に重要なことである.また,
このことはソフトウェア設計をする際にも極めて本質的な事項として浮かび上がってくる.その点については本書の残りの部分を読み進めることで
次第に実感するであろう.

\subsubsection{元気の源}
% 電源,電源IC

最後に縁の下の力持ちを紹介して役者紹介の締めくくりとしよう.今まで CanSat の電装に必要な部品を見てきたが,重要なことを忘れている.
それは電源である.電源がなければコンピュータを起動することも演算をすることも通信をすることもできない.電源は CanSat にとって (当然
他の電装にとっても) なくてはならないものである.

電源を構成するのに必要なものは主にバッテリーと電圧を調整する電源 IC と呼ばれる部品である.いずれの部品も選定を間違えれば
深刻な問題を引き起こすので慎重に選定する必要がある (選定については\ref{sub:部品の選び方}節を参照).バッテリーにはいくつか種類があり,
CanSat 製作時によく使われるバッテリーを表\ref{tab:2.2.4-battery}に示す.

\begin{table}[htbp]
  \centering
  \caption{CanSat でよく用いるバッテリー}
  \begin{tabular}{c|r} \hline
    バッテリー & 概要 \\ \hline
  \end{tabular}
  \label{tab:2.2.4-battery}
\end{table}

バッテリーと同時に検討されるべきなのが電源 IC である.電源 IC には様々な種類があるが,基本的な用途は降圧\footnote{電圧を下げること.}
である場合が多い.降圧が出来る電源 IC の代表格は3端子レギュレータやDC/DCコンバータ\footnote{スイッチングレギュレータとも言う.}がある.
3端子レギュレータは文字通り端子が3つ存在する IC で,各端子は入力,出力,GND 用である.入力端子に電圧を印加すると決まった電圧に降圧し
出力端子に出力されるのが特徴である.3端子レギュレータは入力端子と出力端子の差分である電圧によって発生する電気エネルギーが
熱エネルギーに変換されて放出される.このようなことから後に述べるDC/DCコンバータに比べて変換効率が劣っており,発熱対策が必要となる
場合もある.ただし長所として回路が単純であり,DC/DCコンバータで発生するようなスイッチングノイズがないことが利点である.
一方でDC/DCコンバータは入力電圧を所望の出力電圧に変換するために,電力をスイッチングする (つまり付けたり切ったりする).このように
スイッチングした電圧波形は矩形波の形をしている.そしてこの矩形波を電圧波形をコイルとダイオードを用いて整流\footnote{交流のような時間的に
変動する電圧を直流電圧のように定常的な電圧信号に変換すること.}する.これにより,3端子レギュレータに比べ降圧による電力消費を
押さえながら効率よく降圧を行うことが出来る.ただしDC/DCコンバータの周辺回路は3端子レギュレータの周辺回路に比べ複雑であり,
基板の空間的コストが生じる.また,スイッチングを行うため出力電圧にノイズが発生する可能性があり,ノイズ対策を怠ると部品が破損する可能性がある.

電源は回路において非常に重要であり,最も慎重に組むべき回路である.だがその分理解すべきことも多くあり,ちゃんとやろうと気負うと
容易にノイローゼになってしまう分野でもある.電源回路にあまり慣れていないうちは,そういうものなのかと受け入れ,先人たちの回路を
丸パクリしたほうが身のためである.ただ,精神的余裕のあるときに少しづつ深めていくと非常に面白い分野であることに気づくだろう.
具体的な部品選定の方法や回路設計の方法は後に再度述べるので,今はこのくらいにしておこう.

\subsection{データシート}
% データシートの収集,その重要性


\subsection{コミュニティの形成}
% 部品間の通信 -> 電圧

\subsubsection{情報の通る道路}
% 配線

\subsubsection{コミュニケーションの方法}
% 通信プロトコル


\subsection{部品の選び方}
\label{sub:部品の選び方}

\subsubsection{使いたいもの・必要なもの}

\subsubsection{電力設計は大事}


\subsection{まとめ}

\clearpage
%%%%%%%%%%%%%%%%%%%%%%%%%%%%%%%%%%%%%%%%%%%%%%%%%%%%%%%%%%%%%%%%%%%%%%%%%%%%%%%%%%%%%
%%%%%%%%%%%%%%%%%%%%%%%%%%%%%%%%%% SEC 3 %%%%%%%%%%%%%%%%%%%%%%%%%%%%%%%%%%%%%%%%%%%%
%%%%%%%%%%%%%%%%%%%%%%%%%%%%%%%%%%%%%%%%%%%%%%%%%%%%%%%%%%%%%%%%%%%%%%%%%%%%%%%%%%%%%

\section{全体を俯瞰してみる}

\subsection{CanSat でやりたいこと}
% 目的の確認

\subsection{コミュニティを掌握する}
% 目的達成のために部品をどう操るか

\subsection{少しづつ具体的に}
% 方向性が決まった具体的にどうするか考える

\subsection{まとめ}

\clearpage
%%%%%%%%%%%%%%%%%%%%%%%%%%%%%%%%%%%%%%%%%%%%%%%%%%%%%%%%%%%%%%%%%%%%%%%%%%%%%%%%%%%%%
%%%%%%%%%%%%%%%%%%%%%%%%%%%%%%%%%% SEC 4 %%%%%%%%%%%%%%%%%%%%%%%%%%%%%%%%%%%%%%%%%%%%
%%%%%%%%%%%%%%%%%%%%%%%%%%%%%%%%%%%%%%%%%%%%%%%%%%%%%%%%%%%%%%%%%%%%%%%%%%%%%%%%%%%%%

\section{ハードウェアの製作}

\subsection{目的の確認}

\subsection{システムブロック}
% システムブロックとは (説明とその効果)
% 何を使って書くか
% システムブロックを書くときのポイント
% 画像化

\subsection{EAGLE}

\subsubsection{EAGLE ってなにそれ美味しいの?}

\subsubsection{ファイル}
% Schematic, Board, Library, Design Block

\subsubsection{参考文献・参考サイト}


\subsection{回路の設計}

\subsubsection{分割していこう}
% システムブロックを見ながら回路を分割
% 分割する上でのおすすめの回路の書き方

\subsubsection{地味なことほど大事}
% パターン幅,クリアランスの設定
% Frame
% 線が重なっていないか


\subsection{基板の設計}

\subsubsection{外見には気をつけろ}
% 基板設計で一番大事なのはメカとの整合

\subsubsection{いつも心にインターフェース}

\subsubsection{出来るだけシンプルに}

\subsubsection{仕上げ}


\subsection{基板の実装}

\subsubsection{はんだづけ}

\subsubsection{表面実装}

\subsubsection{通電・電圧確認は忘れずに}


\subsection{まとめ}

\clearpage
%%%%%%%%%%%%%%%%%%%%%%%%%%%%%%%%%%%%%%%%%%%%%%%%%%%%%%%%%%%%%%%%%%%%%%%%%%%%%%%%%%%%%
%%%%%%%%%%%%%%%%%%%%%%%%%%%%%%%%%% SEC 5 %%%%%%%%%%%%%%%%%%%%%%%%%%%%%%%%%%%%%%%%%%%%
%%%%%%%%%%%%%%%%%%%%%%%%%%%%%%%%%%%%%%%%%%%%%%%%%%%%%%%%%%%%%%%%%%%%%%%%%%%%%%%%%%%%%

\section{ソフトウェアの製作}

\subsection{目的の確認}

\subsubsection{何のために演算をするのか}
% 2章のおさらい

\subsubsection{演算をする手段}
% コードを書く


\subsection{設計が先でコードは後}

\subsubsection{ソフトウェアの設計とは何なのか}

\subsubsection{抽象化の威力}
% 各コンポーネントの抽象化について

\subsubsection{どう支配するか}
% 全体のシステム設計


\subsection{知識の暴力}
% ソフトウェアを製作する上で必要な知識の説明


\subsection{まとめ}


\clearpage
%%%%%%%%%%%%%%%%%%%%%%%%%%%%%%%%%%%%%%%%%%%%%%%%%%%%%%%%%%%%%%%%%%%%%%%%%%%%%%%%%%%%%
%%%%%%%%%%%%%%%%%%%%%%%%%%%%%%%%%% SEC 6 %%%%%%%%%%%%%%%%%%%%%%%%%%%%%%%%%%%%%%%%%%%%
%%%%%%%%%%%%%%%%%%%%%%%%%%%%%%%%%%%%%%%%%%%%%%%%%%%%%%%%%%%%%%%%%%%%%%%%%%%%%%%%%%%%%

\section{制御}

\subsection{アイディアを描いてみる}
% フローチャート

\subsection{アイディアをプログラムしてみる}
% どのようにコードに落とし込むか

\subsection{ご機嫌はいかが?}
% 実験とフィードバック

\subsection{研究第一主義}
% 制御についてやるべきこと

\subsection{まとめ}


\clearpage
%%%%%%%%%%%%%%%%%%%%%%%%%%%%%%%%%%%%%%%%%%%%%%%%%%%%%%%%%%%%%%%%%%%%%%%%%%%%%%%%%%%%%
%%%%%%%%%%%%%%%%%%%%%%%%%%%%%%%%%% SEC 7 %%%%%%%%%%%%%%%%%%%%%%%%%%%%%%%%%%%%%%%%%%%%
%%%%%%%%%%%%%%%%%%%%%%%%%%%%%%%%%%%%%%%%%%%%%%%%%%%%%%%%%%%%%%%%%%%%%%%%%%%%%%%%%%%%%

\section{設計思想}

\subsection{思想は便利で鈍感}
% 設計思想を持つことのメリットとデメリット

\subsection{常に抽象的であれ}
% 常に俯瞰的であることについて

\subsection{先人に学ぶ}
% 他人の方法から学ぶ

\subsection{まとめ}


\clearpage
%%%%%%%%%%%%%%%%%%%%%%%%%%%%%%%%%%%%%%%%%%%%%%%%%%%%%%%%%%%%%%%%%%%%%%%%%%%%%%%%%%%%%
%%%%%%%%%%%%%%%%%%%%%%%%%%%%%%%%%% SEC 8 %%%%%%%%%%%%%%%%%%%%%%%%%%%%%%%%%%%%%%%%%%%%
%%%%%%%%%%%%%%%%%%%%%%%%%%%%%%%%%%%%%%%%%%%%%%%%%%%%%%%%%%%%%%%%%%%%%%%%%%%%%%%%%%%%%

\section{立ち止まって振り返る}

\subsection{CanSat の電装専門など要らない}
% 電装の知識は CanSat に留まるものではないという主張

\subsection{数手先を読めるか}
% 完成形を思い浮かべながら設計を行うべきであることを強調

\subsection{あとはやるだけ}
% 頭の中だけでなく実際に手を動かすことも大事であることを強調

\subsection{まとめ}


\clearpage
%%%%%%%%%%%%%%%%%%%%%%%%%%%%%%%%%%%%%%%%%%%%%%%%%%%%%%%%%%%%%%%%%%%%%%%%%%%%%%%%%%%%%
%%%%%%%%%%%%%%%%%%%%%%%%%%%%%%%%%% SEC 9 %%%%%%%%%%%%%%%%%%%%%%%%%%%%%%%%%%%%%%%%%%%%
%%%%%%%%%%%%%%%%%%%%%%%%%%%%%%%%%%%%%%%%%%%%%%%%%%%%%%%%%%%%%%%%%%%%%%%%%%%%%%%%%%%%%

\section{この先にあるもの}

\subsection{標準化}
% 規格を定める

\subsection{高速化}
% 言語,アルゴリズム

\subsection{ハードウェア開拓}
% 自作アナログ回路,インターフェースの追求


\subsection{スケーラビリティを意識する}

\subsubsection{水平スケール}

\subsubsection{垂直スケール}

\subsubsection{ネットワーク}


\subsection{チャレンジ}
% チャレンジングなミッションの提案

\clearpage
%%%%%%%%%%%%%%%%%%%%%%%%%%%%%%%%%%%%%%%%%%%%%%%%%%%%%%%%%%%%%%%%%%%%%%%%%%%%%%%%%%%%%
%%%%%%%%%%%%%%%%%%%%%%%%%%%%%%%%%% SEC 10 %%%%%%%%%%%%%%%%%%%%%%%%%%%%%%%%%%%%%%%%%%%
%%%%%%%%%%%%%%%%%%%%%%%%%%%%%%%%%%%%%%%%%%%%%%%%%%%%%%%%%%%%%%%%%%%%%%%%%%%%%%%%%%%%%

\section{最後に}


\clearpage
\begin{thebibliography}{9}
  % \bibitem{}
\end{thebibliography}


\end{document}
